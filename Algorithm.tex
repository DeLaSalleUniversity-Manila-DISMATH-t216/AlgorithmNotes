\documentclass[journal]{IEEEtran}

\usepackage{algorithmic}
\usepackage{algorithm}
\usepackage{amssymb} % mathbb

\usepackage{xcolor}

\usepackage{mathtools} %floor
\DeclarePairedDelimiter\ceil{\lceil}{\rceil}
\DeclarePairedDelimiter\floor{\lfloor}{\rfloor}

\usepackage[nolists,nomarkers,tablesfirst]{endfloat}

% Redefine the rule colors
% WARNING: may have adverse effects on other float rule color (untested)
\makeatletter
\renewcommand\fs@ruled{\def\@fs@cfont{\bfseries}\let\@fs@capt\floatc@ruled
  \def\@fs@pre{{\color{red}\hrule height.8pt depth0pt \kern2pt}}%
  \def\@fs@post{{\color{red}\kern2pt\hrule\relax}}%
  \def\@fs@mid{{\kern2pt\color{red}\hrule\kern2pt}}%
  \let\@fs@iftopcapt\iftrue}
\makeatother

\definecolor{forestgreen}{rgb}{0.13, 0.55, 0.13}
\definecolor{green(html/cssgreen)}{rgb}{0.0, 0.5, 0.0}


\begin{document}

\title{A Compendium of Basic Algorithms}

\author{\IEEEauthorblockN{Melvin Cabatuan}\\
\IEEEauthorblockA{Electronics and Computer Engineering Department\\
De La Salle University\\
Manila, Philippines\\
Email: melvin.cabatuan@dlsu.edu.ph}}

% make the title area
\maketitle

\section{Introduction}
This lecture notes includes a collection of commonly
used algorithms in an introductory course for a computational background.
An algorithm is a finite set of precise instructions 
for performing a computation or for solving a problem.
An algorithm has input values from a specified set.
From each set of input values, an algorithm produces output values.
An algorithm should yield the correct output values for each set of input values.
Furthermore, the steps of an algorithm must be defined precisely and the desired
output must be produced after a finite number of steps.
Finally, the procedure should be applicable for all problems of the desired form, not just for a
particular set of input.

An algorithm is often described using a pseudocode which is a high-level description of an algorithm that uses the structural conventions of a standard programming language, but is intended for human reading. The following sections describe an algorithm in terms of a pseudocodes.


\section{Pseudocodes}
 
 

\floatname{algorithm}{\color{green(html/cssgreen)}
\centering\textbf{Finding the Maximum}\\
Algorithm}
\begin{algorithm}
  \caption{\color{green(html/cssgreen)} Calculate $maximum$ element in a finite integer set}
  \color{blue}
  \begin{algorithmic}
      \REQUIRE $\{a_1, a_2,\ldots, a_i, \ldots, a_n\} \in \mathbb{Z}$
  \ENSURE $result = max({a_1, a_2,\ldots, a_i, \ldots, a_n})$

  \STATE $result \leftarrow a_1$
  \FOR{$i=2$ to $n$}
      \IF{$result < a_i$}
      \STATE $result \leftarrow a_i$
      \ENDIF
  \ENDFOR

  \end{algorithmic}
\end{algorithm}



\floatname{algorithm}{\color{green(html/cssgreen)}
\centering\textbf{Finding the Minimum}\\
Algorithm}
\begin{algorithm}
  \caption{\color{green(html/cssgreen)}Calculate $minimum$ element in a finite integer set}
  \color{blue}
  \begin{algorithmic}
      \REQUIRE $\{a_1, a_2,\ldots, a_i, \ldots, a_n\} \in \mathbb{Z}$
  \ENSURE $result = min({a_1, a_2,\ldots, a_i, \ldots, a_n})$

  \STATE $result \leftarrow a_1$
  \FOR{$i=2$ to $n$}
      \IF{$result > a_i$}
      \STATE $result \leftarrow a_i$
      \ENDIF
  \ENDFOR

  \end{algorithmic}
\end{algorithm}


 
 
\floatname{algorithm}{\color{green(html/cssgreen)}
\centering\textbf{Linear/Sequential Search (Using a For-Loop)}\\
Algorithm}
\begin{algorithm}
  \caption{\color{green(html/cssgreen)} Locate an element $x$ in a list of distinct
values or determine that it is not in the list.}
  \color{blue}
  \begin{algorithmic}
      \REQUIRE $\{a_1, a_2,\ldots, a_i, \ldots, a_n\}_{\ne} \in \mathbb{Z}$; $x \in \mathbb{Z}$
  \ENSURE $result = k $, where $ (a_k = x) $ and $ k \in \{1,\ldots,n\} $ if the element is found; otherwise $k = -1$ 

  \STATE $result \leftarrow -1$
  \FOR{$i=1$ to $n$}
      \IF{$result == a_i$}
      \STATE $result \leftarrow i$
      \ENDIF
  \ENDFOR

  \end{algorithmic}
\end{algorithm}
 

 
 
 \floatname{algorithm}{\color{green(html/cssgreen)}
 \centering\textbf{Linear/Sequential Search (Using a While-Loop)}\\
 Algorithm}
\begin{algorithm}
  \caption{\color{green(html/cssgreen)}Locate an element $x$ in a list of distinct
values or determine that it is not in the list.}
  \color{blue}
  \begin{algorithmic}
      \REQUIRE $\{a_1, a_2,\ldots, a_i, \ldots, a_n\}_{\ne} \in \mathbb{Z}$; $x \in \mathbb{Z}$
  \ENSURE $result = k $, where $ (a_k = x) $ and $ k \in \{1,\ldots,n\} $ if the element is found; otherwise $k = -1$ 

  \STATE $i \leftarrow 1$
  
  \WHILE{$(i \le n) \wedge (x \ne a_i) $}
  	  \STATE $i \leftarrow i + 1$
  \ENDWHILE
  
  \IF{$i \le n$}
      \STATE $result \leftarrow i$
  \ELSE 
      \STATE $result \leftarrow -1$
  \ENDIF

  \end{algorithmic}
\end{algorithm}



 
 \floatname{algorithm}{\color{green(html/cssgreen)}
 \centering\textbf{Binary Search}\\
 Algorithm}
\begin{algorithm}
  \caption{\color{green(html/cssgreen)}Locate an element $x$ in a list of distinct
and sorted values or determine that it is not in the list.}
  \color{blue}
  \begin{algorithmic}
      \REQUIRE $\{a_1, a_2,\ldots, a_i, \ldots, a_n\}_{\ne} \in \mathbb{Z}$, where $a_1<a_2<\ldots<a_n$; $x \in \mathbb{Z}$
  \ENSURE $result = k $, where $ (a_k = x) $ and $ k \in \{1,\ldots,n\} $ if the element is found; otherwise $k = -1$ 

  \STATE $i \leftarrow 1$
  \STATE $j \leftarrow n$
  
  \WHILE{$i < j$}
  	  \STATE $mid \leftarrow \floor*{\dfrac{i + j}{2}}$
  	  \IF{$x > a_{mid}$}
         \STATE $i \leftarrow mid + 1$
      \ELSE 
         \STATE $j \leftarrow mid$
      \ENDIF
  \ENDWHILE
  
  \IF{$x == a_i$}
      \STATE $result \leftarrow i$
  \ELSE 
      \STATE $result \leftarrow -1$
  \ENDIF

  \end{algorithmic}
\end{algorithm}
 
 
 
 \floatname{algorithm}{\color{green(html/cssgreen)}
 \centering\textbf{Bubble Sort}\\
 Algorithm}
\begin{algorithm}
  \caption{\color{green(html/cssgreen)} Given a list of elements of a set, sort in increasing order.}
  \color{blue}
  \begin{algorithmic}
      \REQUIRE $\{a_1, a_2,\ldots, a_i, \ldots, a_n\} \in \mathbb{R}$, $n \ge 2$
  \ENSURE $\{a_1, a_2,\ldots, a_n\}$ where $a_1 < a_2 < \ldots < a_n$
   
  \FOR{$i=1$ to $n-1$}
      \FOR{$j=1$ to $n-i$}
         \IF{$a_j > a_{j+1}$}
            \STATE $temp \leftarrow a_{j+1}$
            \STATE $a_{j+1} \leftarrow a_j$
            \STATE $a_j \leftarrow temp$
         \ENDIF
      \ENDFOR
  \ENDFOR

  \end{algorithmic}
\end{algorithm}




 \floatname{algorithm}{\color{green(html/cssgreen)}
 \centering\textbf{Insert Sort}\\
 Algorithm}
\begin{algorithm}
  \caption{\color{green(html/cssgreen)} Given a list of elements of a set, sort in increasing order.}
  \color{blue}
  \begin{algorithmic}
      \REQUIRE $\{a_1, a_2,\ldots, a_i, \ldots, a_n\} \in \mathbb{R}$, $n \ge 2$
  \ENSURE $\{a_1, a_2,\ldots, a_n\}$ where $a_1 < a_2 < \ldots < a_n$
   
  \FOR{$j=2$ to $n$}
     \STATE $i \leftarrow 1$
      \WHILE{$a_j > a_i$}
           \STATE $i \leftarrow i + 1$
      \ENDWHILE
           \STATE $temp \leftarrow a_j$
           \FOR{$k=0$ to $j-i-1$}
               \STATE $a_{j-k} = a_{j-k-1}$
           \ENDFOR
           \STATE $a_i = temp$
  \ENDFOR

  \end{algorithmic}
\end{algorithm}



 
 
\floatname{algorithm}{\color{green(html/cssgreen)}
\centering\textbf{Greedy Change-Making}\\
Algorithm}
\begin{algorithm}
  \caption{\color{green(html/cssgreen)}Make $x$ cents change with quarters, dimes, nickels, and pennies, and
using the least total number of coins.}
  \color{blue}
  \begin{algorithmic}
      \REQUIRE $\{c_1, c_2,\ldots, c_i, \ldots, c_n\} \in \mathbb{C}$oin Denomination \\
          where $c_1 > c_2 > \ldots > c_n$; $x$ - amount of money in cents
  \ENSURE $result = $ minimum number of coins

  \STATE $result \leftarrow 0$
  \FOR{$i=1$ to $n$}
  	 \WHILE{$x \ge c_i$}
  	   \STATE $x \leftarrow x - c_i$ 
       \STATE $result \leftarrow result + 1$
      \ENDWHILE
  \ENDFOR

  \end{algorithmic}
\end{algorithm}
 
 
 
\floatname{algorithm}{\color{green(html/cssgreen)}
\centering\textbf{Greedy Change-Making per Denomination}\\
Algorithm}
\begin{algorithm}
  \caption{\color{green(html/cssgreen)}Make $x$ cents change with quarters, dimes, nickels, and pennies, and
using the least total number of coins.}
  \color{blue}
  \begin{algorithmic}
      \REQUIRE $\{c_1, c_2,\ldots, c_i, \ldots, c_n\} \in \mathbb{C}$oin Denomination \\
          where $c_1 < c_2 < \ldots < c_n$; $x$ - amount of money in cents
  \ENSURE $result[n] = $ minimum number of coins per denomination

  \FOR{$i=1$ to $n$}
     \STATE $result_i \leftarrow 0$
  	 \WHILE{$x \ge c_i$}
  	   \STATE $x \leftarrow x - c_i$ 
       \STATE $result_i \leftarrow result_i + 1$
      \ENDWHILE
  \ENDFOR

  \end{algorithmic}
\end{algorithm}
 
 
 
 
 \floatname{algorithm}{\color{green(html/cssgreen)}
\centering\textbf{Greedy Algorithm for Scheduling Talks}\\
Algorithm}
\begin{algorithm}
  \caption{\color{green(html/cssgreen)}Suppose we have a group of proposed talks with preset start and end times. Schedule as many of these talks as possible in a lecture hall, under the assumptions that once a talk starts, it continues until it ends, no two talks can proceed at the same time, and a talk can begin at the same time another one ends.}
  \color{blue}
  \begin{algorithmic}
      \REQUIRE schedule $\{(s_1, f_1), (s_2, f_2),\ldots, (s_n, f_n)\}$ where \\ $s_i$ - start times of talks, and $f_i$ - finish times of talks            
  \ENSURE $result = $ set of scheduled talks

  \STATE $result \leftarrow \emptyset$
  \STATE $sort(\{(s_1, f_1), (s_2, f_2),\ldots, (s_n, f_n)\})$ w.r.t. finish times   
  \FOR{$i=1$ to $n$}
      \IF{$(s_i, f_i)$ is compatible with $result$}
      \STATE $result \leftarrow  result \cup (s_i, f_i)$
      \ENDIF
  \ENDFOR

  \end{algorithmic}
\end{algorithm}
 
 
 
\floatname{algorithm}{\color{green(html/cssgreen)}
   \centering\textbf{Matrix Multiplication}\\
   Algorithm}
\begin{algorithm}
  \caption{\color{green(html/cssgreen)}Multiply an $m \times k$ matrix by a $k \times n$ matrix.}
  \color{blue}
  \begin{algorithmic}
      \REQUIRE $m \times p$ matrix $A = [a_{ij}]$; $p \times n$ matrix $B = [b_{ij}]$
  \ENSURE $result = m \times n$ matrix $C = [c_{ij}]$

  \FOR{$i=1$ to $m$}
		 \FOR{$i=1$ to $n$}
		 	  \STATE $c_{ij} \leftarrow 0$
		      \FOR{$k=1$ to $p$}
                   \STATE $c_{ij} \leftarrow c_{ij} + a_{ik}b_{kj}$
  		 	  \ENDFOR
  		 \ENDFOR
  \ENDFOR

  \end{algorithmic}
\end{algorithm}
 
 
 
 \floatname{algorithm}{\color{green(html/cssgreen)}
   \centering\textbf{Boolean Product of Zero-One Matrices}\\
   Algorithm}
\begin{algorithm}
  \caption{\color{green(html/cssgreen)}Calculate the Boolean product of an $m \times k$ zero-one matrix by a $k \times n$ zero-one matrix.}
  \color{blue}
  \begin{algorithmic}
      \REQUIRE $m \times p$ matrix $A = [a_{ij}]$; $p \times n$ matrix $B = [b_{ij}]$
  \ENSURE $result = m \times n$ matrix $C = [c_{ij}]$

  \FOR{$i=1$ to $m$}
		 \FOR{$i=1$ to $n$}
		 	  \STATE $c_{ij} \leftarrow 0$
		      \FOR{$k=1$ to $p$}
                   \STATE $c_{ij} \leftarrow c_{ij} \vee (a_{ik} \wedge b_{kj})$
  		 	  \ENDFOR
  		 \ENDFOR
  \ENDFOR

  \end{algorithmic}
\end{algorithm}
 
 
 
    \floatname{algorithm}{\color{green(html/cssgreen)}
   \centering\textbf{Closest Pair}\\
   Algorithm}
\begin{algorithm}
  \caption{\color{green(html/cssgreen)}Find the closest pair of points by computing the distances between all pairs of the $n$ points and determining the smallest distance.}
  \color{blue}
  \begin{algorithmic}
      \REQUIRE $\{(x_1, y_1), (x_2, y_2), \ldots, (x_n, y_n)\}$ where $x_k, y_k \in \mathbb{R}$
  \ENSURE $result = minimum(distance((x_i, y_i), (x_j, y_j)))$

  \STATE $min \leftarrow \infty$   
  \FOR{$i=2$ to $n$}
     \FOR{$j=1$ to $i-1$}
        \IF{$(x_j - x_i)^2 + (y_j - y_i)^2 < min$}
            \STATE $min \leftarrow (x_j - x_i)^2 + (y_j - y_i)^2$
            \STATE $result \leftarrow ((x_i, y_i), (x_j, y_j))$
        \ENDIF
     \ENDFOR
  \ENDFOR

  \end{algorithmic}
\end{algorithm}
 

\clearpage
 
 \begin{thebibliography}{1}

\bibitem{Algorithm:rosen}
K.~Rosen, \emph{Discrete mathematics and its applications}, 7th~ed.\hskip 1em plus
  0.5em minus 0.4em\relax New York, NY: McGraw-Hill, 2012.

\end{thebibliography}

\end{document}
